\chapter{benchmark frameworks php}
  \section*{CakePHP}
    Comme son nom l’indique, CakePHP, est un framework PHP. Il est relativement simple à prendre en main comparé à certains autres frameworks PHP tels que Zend, pour ne citer que l’un des plus connus
    CakePHP permet le développement rapide d’applications (RAD : Rapid Application Development) et être « greffés » ceux-ci sont évidemment testés et particulièrement observés par l’équipe en place.
    Bien que n’étant pas aujourd’hui en mesure de concurrencer ses aînés tels Zend ou CakePHP, Jelix possède de sérieux atouts, et compte parmi les meilleurs outils de développement PHP, et notamment pour des architectures à fortes charges.
  \section*{Symfony}
    Conçu dès l’origine en 2005 par une société française, Sensio Labs, bien connue pour ses idées novatrices dans le développement web, Symfony est robuste, fiable, très puissant, et s’adapte à un très grand nombre de configurations.
    Développé par une équipe de professionnels expérimentés, il est spécialement dédié à la conception d’applications moyennes à très lourdes, c’est pourquoi il trouve tout naturellement sa place dans 

\chapter{Modèle de test}
  SOLID (Single responsibility, Open-closed, Liskov substitution, Interface segregation and Dependency inversion) est un acronyme introduit par Michael Feathers pour les «cinq principes de base» nommés par Robert C. Martin au début des années 2000. Ces 5 principes sont censés apporter une ligne directrice permettant la conception et le développement de logiciel orientée objet d'une manière plus fiable et plus robuste. Les principes mis ensemble, ont l'intention de rendre plus probable que le programmeur 