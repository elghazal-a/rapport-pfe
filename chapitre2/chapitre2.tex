\chapter{Le cloud computing}
\begin{onehalfspace}

\initial{L} e but de ce chapitre est d'exposer le travail fait au niveau de la documentation. Ce chapitre est très riche par des concepts incontournables à la suite de ce rapport. En effet, nous allons commencer par expliquer ce que signifie le cloud concrètement. Ensuite nous présenterons les types de cloud les plus classiques, à savoir IaaS, PaaS et SaaS. Enfin, les notions machines virtuelles et conteneurs seront présentées avec une comparaison détaillée des deux.


\newpage


\section{Le Cloud Computing}

\subsection{Définition}

Le \textbf{Cloud Computing} est l'exploitation de la puissance de calcul ou de stockage de serveurs informatiques distants par l'intermédiaire du réseau internet. Ces serveurs sont loués à la demande selon des critères technique (Bande passante, puissance, etc.). Le cloud computing se caractérise par sa grande souplesse. En effet, il est déstiné aux utilisateurs de tous les niveaux de compétences.

Le cloud est rendu possible grâce à la virtualisation, l'ubiquité des réseaux à grande vitesse, les capacités des navigateurs d'aujourd'hui et l'évolution des piles de développement Web. Avec ces choses en place, il devient moins nécessaire de posséder votre propre infrastructure, ou même de posséder votre propre logiciel. Vous pouvez obtenir ce que vous avez besoin à partir du Cloud, tant que vous en avez besoin.


\subsection{Caractéristiques}

En termes clairs, le Cloud donne la capacité pour les utilisateurs finaux d'utiliser des pièces de ressources. Ces ressources doivent être acquises rapidement et facilement. NIST définit plusieurs caractéristiques qu'il juge essentiel pour qu'un service soit considéré comme «Cloud». Ces caractéristiques comprennent:

\begin{itemize}

\item Service à la demande. La capacité pour un utilisateur final de s' inscrire et recevoir des services sans les longs délais qui ont caractérisé l'informatique traditionnelle;

\item Accessible au réseau large. La Capacité d'accéder au service via les plates-formes standard (bureau, ordinateur portable, mobiles, etc.);

\item La mise en commun des ressources. Les fournisseurs servent plusieurs clients ou «locataires» avec des services provisoires et scalables. Ces services peuvent être ajustés pour répondre aux besoins de chaque client, sans aucune modification apparente pour l'utilisateur;

\item Rapide élasticité. La capacité des ressources doit évoluer pour faire face aux pics de la demande;

\item Service mesuré. La facturation est mesuré et livré



\end{itemize}


\subsection{Caractéristiques}

\section{Virtualisation ou containérisation}

\section{Docker}

\end{onehalfspace}