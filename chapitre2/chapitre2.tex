\chapter{Le cloud computing}
\begin{onehalfspace}

\initial{L} e but de ce chapitre est d'exposer le travail fait au niveau de la documentation. Ce chapitre est très riche par des concepts incontournables à la suite de ce rapport. En effet, nous allons commencer par expliquer ce que signifie le cloud concrètement. Ensuite nous présenterons les types de cloud les plus classiques, à savoir IaaS, PaaS et SaaS. Enfin, les notions machines virtuelles et conteneurs seront présentées avec une comparaison détaillée des deux.


\newpage


\section{Le Cloud Computing}

\subsection{Définition}

Le \textbf{Cloud Computing} est l'exploitation de la puissance de calcul ou de stockage de serveurs informatiques distants par l'intermédiaire du réseau internet. Ces serveurs sont loués à la demande selon des critères technique (Bande passante, puissance, etc.). Le cloud computing se caractérise par sa grande souplesse. En effet, il est déstiné aux utilisateurs de tous les niveaux de compétences.

Le cloud est rendu possible grâce à la virtualisation, l'ubiquité des réseaux à grande vitesse, les capacités des navigateurs d'aujourd'hui et l'évolution des piles de développement Web. Avec ces choses en place, il devient moins nécessaire de posséder votre propre infrastructure, ou même de posséder votre propre logiciel. Vous pouvez obtenir ce que vous avez besoin à partir du Cloud, tant que vous en avez besoin.


\subsection{Caractéristiques}

En termes clairs, le Cloud donne la capacité pour les utilisateurs finaux d'utiliser des pièces de ressources. Ces ressources doivent être acquises rapidement et facilement. NIST définit plusieurs caractéristiques qu'il juge essentiel pour qu'un service soit considéré comme «Cloud». Ces caractéristiques comprennent:

\begin{itemize}

\item Service à la demande. La capacité pour un utilisateur final de s' inscrire et recevoir des services sans les longs délais qui ont caractérisé l'informatique traditionnelle;

\item Accessible au réseau large. La Capacité d'accéder au service via les plates-formes standard (bureau, ordinateur portable, mobiles, etc.);

\item La mise en commun des ressources. Les fournisseurs servent plusieurs clients ou «locataires» avec des services provisoires et scalables. Ces services peuvent être ajustés pour répondre aux besoins de chaque client, sans aucune modification apparente pour l'utilisateur;

\item Rapide élasticité. La capacité des ressources doit évoluer pour faire face aux pics de la demande;

\item Service mesuré. La facturation est mesuré et livré


\end{itemize}

Sans ces caractéristiques, l'informatique en nuage n'appore rien par rapport à l'informatique traditionnelle. Une solution Cloud doit démontrer ces caractéristiques et notre projet ne doit pas échapper à cette règle. Dans ce qui suit, nous détaillerons les types classiques du cloud.

\subsection{Les types de Cloud}

Le Cloud Computing est un large terme qui décrit une large collection de services. Dans ce rapport, nous allons expliquer les différents types de services de Cloud communément appelé Software as a Service (SaaS),  Platform as a Service (PaaS) et Infrastructure as a Service (IaaS) et décrirent plus en détails ceux qui touchent à notre sujet.

\begin{figure}[H]
\centering
\includegraphics [scale=0.7]{chapitre2/assets/pyramide.jpg}
\caption{Pyramide des services Cloud}
\end{figure}

\subsubsection*{Software as a Service}

La meilleure façon de comprendre ces services est de commencer avec le SaaS, la couche la plus abstraite et celui qu'on utilise peut-être déjà aujourd'hui, même à un niveau personnel. Un exemple simple de SaaS est un service de messagerie en ligne, comme Gmail. Lorsque l'on utilise Gmail, vous n'êtes pas hébergez votre propre serveur de messagerie. C'est Google qui l'héberge, et on est tout simplement entrain d'y accéder via un navigateur comme un client.

SaaS est vraiment orienté vers les utilisateurs finaux de l'entreprise et ne nécéssite pas beaucoup de compétences pour l'utiliser. Le fournisseur décide sur le nombre de ressources à consacrer à l'utilisation de l'application. Le fournisseur décide sur les serveurs, les machines virtuelles, l'équipement de réseau, tout. Enfin, il suffit de pointer le navigateur à l'application.

\subsubsection*{Infrastructure as a Service}

IaaS est à l'autre bout du pyramide du Cloud. Lorsque l'on souhaite garder le contrôle de l'environnement logiciel, mais on veut pas maintenir aucun équipements; Lorsque l'on veut pas acheter des serveurs et de les mettre dans une pièce à température contrôlée ou rien de tout cela; On va chez un fournisseur IaaS et demander tout simplement une machine virtuelle.

L'on peut mettre n'importe quel logiciel que l'on souhaite au-dessus d'IaaS. Sur l'arrière, le fournisseur obtient vous de stockage ou d'autres ressources que l'on a besoin. Ceci est rendu plus facile avec les technologies de virtualisation, qui séparent les ressources physiques de la machine virtuelle qui éxécute le logiciel. IaaS est disponible sur Amazon EC2, GCE de Google et bien d'autres.

L'infrastructure en tant que service ou l'IaaS ne touche pas directement à notre sujet. Du coup, il ne sera mentionné que rarement par la suite.

\subsubsection*{Platform as a Service}

PaaS se situe quelque part entre IaaS et SaaS. Il est pas un produit fini, comme SaaS, encore moins une simple ressource virtuelle vierge, comme IaaS. PaaS est destiné pour les développeurs, il leur donne des outils et des interfaces de haut niveau pour y développer dessus. Par exemple, Windows Azure de Microsoft vous donne des outils pour développer des applications mobiles, des applications sociales, sites Web, jeux et plus encore. Vous construisez ces choses, mais vous utilisez les API et les outils pour les accrocher dans l'environnement Azure et de les exécuter là.


\begin{figure}[H]
\centering
\includegraphics [scale=0.5]{chapitre2/assets/cloud-vs.png}
\caption{L'éxternalisation de l'informatique en Cloud}
\end{figure}


\section{Virtualisation ou containérisation}



\subsection{Virtualisation}

Avec des racines profondément ancrées dans l’informatique, la virtualisation sert à partitionner un seul serveur physique en plusieurs machines virtuelles, comme un espace de stockage ou un réseau, en plusieurs ressources virtuelles. Elle permet une consolidation de serveurs avec une grande souplesse d'utilisation. Dans le contexte de l’informatique en nuage, la virtualisation est importante pour la mise en service et le retrait rapide de serveurs. Un logiciel de virtualisation du Cloud présente également une perspective dynamique et une vue unifiée de l’utilisation et de l’efficacité des ressources, cela afin d’assurer le fonctionnement des services du Cloud. La virtualisation est la principale technologie permettant d’arriver à une utilisation rentable des serveurs tout en prenant en charge la séparation entre de multiples locataires d’un matériel physique. Il existe d’autres solutions pour arriver à ces objectifs, mais ses avantages en font l’approche de choix.

\subsection{Containérisation}


\subsection{Etude comparative}

\section{Docker}

\end{onehalfspace}