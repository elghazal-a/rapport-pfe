\chapter{Le cloud computing}
\begin{onehalfspace}

\initial{L} e but de ce chapitre est d'exposer le travail fait au niveau de la documentation. Ce chapitre est très riche par des concepts incontournables à la suite de ce rapport. En effet, nous allons commencer par expliquer ce que signifie le cloud concrètement. Ensuite nous présenterons les types de cloud les plus classiques, à savoir IaaS, PaaS et SaaS. Enfin, les notions machines virtuelles et conteneurs seront présentées avec une comparaison détaillée des deux.


\newpage


\section{Le Cloud Computing}

\subsection{Définition}

Le \textbf{Cloud Computing} est l'exploitation de la puissance de calcul ou de stockage de serveurs informatiques distants par l'intermédiaire du réseau internet. Ces serveurs sont loués à la demande selon des critères technique (Bande passante, puissance, etc.). Le cloud computing se caractérise par sa grande souplesse. En effet, il est déstiné aux utilisateurs de tous les niveaux de compétences.



\section{Virtualisation ou containérisation}

\section{Docker}

\end{onehalfspace}