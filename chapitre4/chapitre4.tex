\chapter{Conception}
\begin{onehalfspace}

\initial{C}e chapitre a pour objectif de présenter la conception du projet. Cette phase est un point de fusion entre l'analyse des besoins et l'étude approfondie de la documentation. Dans un premier temps, une problématique des bases de données sera soulevée. Ensuite, nous allons expliquer comment un \acrshort{saas} doit être conçu. Enfin, une architecture Cloud pour porter les conteneurs sera présentée.

\newpage

\section{Base de donnée}



Odoo, comme toute autre application, a besoin d'une base de donnée \emph{PostgreSQL} pour persister des informations. Malheureusement, la communauté de Docker n'est pas si satisfaite vis-à-vis l'idée de containériser les bases de données. Nous allons prouver à travers deux scénarios qu'il serait une imprudence de tourner les bases de données dans des conteneurs.

Le fait de la rapidité et la légèreté de Docker et des conteneurs en général a poussé CoreOS d'adopter une certaine philosophie à l'égard de la gestion et l'ordonnancement des conteneurs. En effet, lorsqu'un conteneur s'est arrêté pour une raison quelconque, il va être automatiquement écrasé et remplacé par un nouveau qui va tourner le même service. Une philosophie, certes, moins prudente et naïve, mais sa simplicité épargne beaucoup de problèmes à tout le monde.

Par conséquent, on ne peut en aucun cas mettre les données dans le système de fichiers d'un conteneur sous peine de les perdre. La figure \ref{fig:database1} montre qu'après avoir tuer un conteneur, ses données sont perdues.

\begin{figure}[H]
\centering
\includegraphics [scale=0.5]{chapitre4/assets/database1}
\caption{Données perdues - scénario 1}
\label{fig:database1}
\end{figure}

Pour y remédier, l'on peut penser à monter le volume des données dans le serveur hôte, ainsi les données sont plus ou moins persistantes. Outre le fait que les données seront certainement perdues quand le serveur tombe en panne, le conteneur n'aura plus accès aux données quand il sera réordonnancé. En effet, le caractère imprévisible de l'ordonnanceur \emph{Fleet} fait qu'il n'y a pas de garantie qu'un conteneur réside dans le même serveur. La figure \ref{fig:database2} montre ce scénario.

\begin{figure}[H]
\centering
\includegraphics [scale=0.5]{chapitre4/assets/database2}
\caption{Données perdues - scénario 2}
\label{fig:database2}
\end{figure}


Dans un environnement de production, il est encore tôt pour mettre les bases de donnée et les applications à états (stateful) en général dans des conteneurs. De ce fait, il serait judicieux de les faire sortir de l'architecture à base de conteneurs. En effet, deux solutions se portent à nous, les bases de données vont être mis en place suivant l'informatique classique ou elles vont être consommées comme un service à partir d'une partie tierce Cloud (Délégation de l'informatique). La dernière solution est coûteuse mais il est plus fiable.



\section{Scalabilité et disponibilité}


Parmi les contraintes qui ont poussé Sayoo à migrer vers le Cloud, c'est de garantir une haute disponibilité et une scalabilité infinie.

\index{Disponibilité}
La haute disponibilité fait référence au fait qu'un service est en quelque sorte tolérant à un échec ou à une panne. La disponibilité va être assurée en faisant la redondance des services dans différents serveurs du Cluser.

\index{Scalabilité}
La scalabilité ou l'évolutivité, c'est la capacité d'une application à pouvoir exploiter de nouvelles ressources récemment déployées pour répondre à un pic de charge. En fait, la scalabilité n'est pas apporté par le Cloud, qui apporte l'élasticité, mais c'est l'application elle-même qui l'apporte. Par conséquent, l'idée que toute application déployée dans un milieu élastique Cloud peut être évolutive est fausse.


Pour faire une bonne conception d'un \acrshort{saas} scalable, il faut distinguer d'abord deux types d'applications:

\begin{itemize}
	\item Application \textbf{stateful} ou à états : C'est une application dont les processus ne stockent aucune donnée qui doit persister, même dans un temps aussi petit qu'il soit, dans l'espace mémoire ou le système de fichiers.
	\item Application \textbf{stateless} ou sans états : C'est une application qui stocke tous les données persistantes dans un service de support externe (Base de donnée, queue, mémoire cache, etc.).
\end{itemize}

Une application scalable doit être \textbf{stateless}, plus que cela, il doit respecter les douze-facteurs définies dans l'annexe \ref{annexe:12factors}. Une application scalable suppose qu'aucune donnée mise en cache (mémoire, disque) ne sera disponible dans une requête ultérieure. En effet, les chances sont élevées qu'une future requête sera servie par un processus différent du même service. L'idée c'est qu'une application doit être capable de renvoyer les réponses de façon consistante à partir de n'importe quelle instance exécutant l'application.



La figure \ref{fig:non-scalable} montre un service qui n'est évolutive contrairement au service présenté dans la figure \ref{fig:scalable}.


\begin{figure}[H]
\centering
\includegraphics [scale=0.5]{chapitre4/assets/stateful}
\caption{Service non scalable}
\label{fig:non-scalable}
\end{figure}

\begin{figure}[H]
\centering
\includegraphics [scale=0.5]{chapitre4/assets/stateless}
\caption{Service scalable}
\label{fig:scalable}
\end{figure}


\section{Architecture}

\begin{figure}[H]
\centering
\includegraphics [scale=0.7]{chapitre4/assets/architecture}
\caption{Achitecture de la solution}
\label{fig:architecture}
\end{figure}

L'architecture est très simplifiée car nous avons fait une abstraction du comportement interne des composants de CoreOS. Elle décrit trois processus et implique trois acteurs:


\begin{itemize}
 	\item \textbf{Développeur} : Il développe, personnalise et adapte l'image Odoo de base aux besoins du client. Une fois le développement est fait, il pousse l'image Odoo vers le registre des images de Docker. Cette image est étiquetée par un nom, une version et un environnement (développement, test, production).
 	\item \textbf{Exploitant} : Il exploite l'ordonnanceur Fleet pour déployer l'application que le développeur vient de pousser dans le registre. L'exploitant va demander à l'ordonnanceur, par exemple, de lancer le service en mode haute disponibilité. L'ordonnanceur lance plusieurs instances du service dans des serveurs différents. L'exploitant, remarquant un pic de charge sur ce service, va demander à l'ordonnanceur de lancer une autre instance qui compense la surcharge.
 	\item \textbf{Client} : C'est l'utilisateur final de l'application. Il va saisir l'\acrshort{url} de son application. Son requête va être interceptée par l'équilibreur de charge (Load-Balancer) qui va faire suivre la demande à un des serveurs du Cluster. Quand la requête est arrivée, elle sera interceptée par le proxy de ce serveur qui va la faire suivre à l'instance du service la moins chargée, quitte même à la faire suivre vers une instance d'un autre serveur.
 \end{itemize} 


\section*{Conclusion}
En somme, nous avons prouvé l'incompatibilité des bases de données avec les conteneurs dans une solution fiable. Une architecture qui répond aux exigences su Cloud a été présenté, mais il faut pas la gâcher par une mauvaise conception des applications.

\end{onehalfspace}
