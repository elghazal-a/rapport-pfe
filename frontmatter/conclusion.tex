\chapter*{Conclusion}
\addcontentsline{toc}{chapter}{Conclusion}

\begin{onehalfspace}
\initial
Notre stage de fin d'études consistait à mettre en œuvre une architecture Cloud basée sur des conteneurs. Le but de ce projet est de faciliter et simplifier aux développeurs de \emph{Sayoo} le développement, la personnalisation de nouveaux modules \emph{Odoo} (Open ERP), et faciliter le processus de déploiement continue.
\newline
\newline
Certes, c'est \emph{Odoo} qui a poussé à avoir recours au Cloud, mais la solution réalisée est conçu de façon général pour la provision des \acrshort{saas} légers (containérisés). Toutefois, elle n'est pas complète. En effet, Nous aimerions ajouter un service de mesure et de facturation ainsi que d'améliorer le processus de scalabilité et de l'automatiser.
\newline
\newline
\noindent Par ailleurs, nous suivrons avec attention le projet \emph{Flocker} qui s'intéresse à la containérisation des bases de données dans le but de perfectionner l'orchestration des applications \emph{Stateful}. Nous continuerons à surveiller le projet Deis qui intégrera Kubernetes et Mesos dans un futur proche, ce qui rendra l'architecture encore plus performante et optimale. Enfin, nous développerons des interfaces utilisateurs (UI) afin de simplifier la gestion entière de l'architecture et éviter de passer par les lignes de commandes.  	
\newline
\newline
\end{onehalfspace}