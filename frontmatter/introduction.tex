\chapter*{Introduction}
\addcontentsline{toc}{chapter}{Introduction}

\begin{onehalfspace}
Toutes les entreprises fournissent des services à leurs clients, évidemment avec l’objectif de générer des bénéfices ainsi qu'une gestion efficace des ressources. En effet, l'entreprise doit tracer une stratégie efficace au court, moyen et long terme pour améliorer cette gestion  et promouvoir sa relation avec ses clients et partenaires. Le concept de planification des ressources de l'entreprise (\emph{ERP}) est une réponse à ce besoin  car il met en place un système modulaire englobant l'ensemble du système d'informations (SI) où les données sont mises à jour en temps réel et garantit la piste de l'audit. Cependant les \emph{ERP} ne recouvrent pas entièrement les besoins, c'est pourquoi certaines entreprises se penchent sur le développement personnalisé de nouveaux modules. 
\newline
Dans cette optique, la société Sayoo a décidé de s’investir dans le développement des modules \emph{ERP}. Ce service  a pour objectif de faciliter la gestion et le suivi des ressources de l'entreprise ainsi que l'évolution de leurs relations avec les clients, en automatisant plusieurs processus métier. Dans ce cadre, notre stage de fin d'étude a donc pour objectif de mettre en œuvre une architecture Cloud basée sur les conteneurs qui facilitera à Sayoo la création et la personnalisation des modules \emph{ERP}.
\newline
Le présent mémoire expose les résultats de notre travail durant notre stage. Il comporte cinq chapitres organisés suivant le processus de développement choisi:
\begin{itemize}
\item L’objet du premier chapitre est de cerner le projet dans son contexte général, à travers la présentation de la société \emph{Sayoo}, ensuite décrire la motivation du projet et les objectifs visés. La dernière section du chapitre sera consacrée à la conduite et la planification du projet.
\item le deuxième chapitre est consacré à la définition du Cloud computing et l'explication des différents concepts ainsi que l'introduction de la notion de containérisation. 
\item Le troisième chapitre s'attaque à l'orchestration des applications distribuées qui présentera les différentes technologies permettant une gestion multi-conteneurs ainsi qu'un benchmark à la fin du chapitre.
\item Au niveau du quatrième chapitre, nous expliciterons les phases d’analyse et de conception de l'architecture. Nous exposerons plusieurs architectures intermédiaires pour modéliser les fonctionnalités réalisées.
\item Le cinquième et dernier chapitre décrit d’une façon détaillée, la phase de mise en œuvre du projet, à travers quelques interfaces réalisées.
\end{itemize}

\noindent Le mémoire se termine par une conclusion générale qui présente le bilan du travail réalisé et ses principales perceptives.





\clearpage

\end{onehalfspace}