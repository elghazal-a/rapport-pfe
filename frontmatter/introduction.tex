\chapter*{Introduction}
\addcontentsline{toc}{chapter}{Introduction}

\begin{onehalfspace}
Toutes les entreprises vendent des produits et/ou des services à des clients, évidemment avec l’objectif de générer un chiffre d’affaires et des bénéfices les plus élevés possibles. Le concept du crm est une réponse aux limites du modèle de marketing transactionnel, trop centré sur l’achat et la vente mais ne permettant pas un suivi sur le long terme. Ce concept vise à suivre et à faire évoluer la relation avec le client dans le temps, l’objectif de cette relation étant principalement de fidéliser les clients existants ou de cultiver un vivier de prospects.
\newline
Dans cette optique, la société Thinline a décidé de s’investir dans le développement d'un crm pharmaceutique. Cette plate-forme a pour objectif de faciliter, auprès des laboratoires pharmaceutiques au niveau méditerrané, le suivi et l'évolution de leurs relation avec leurs clients, en automatisant plusieurs processus métier. Dans ce cadre, notre stage de fin d'étude a donc pour objectif de mettre en œuvre cette solution à travers l’étude, la conception et la réalisation de trois modules fondamentaux, à savoir, le module « annuaire», le module « calendrier» et le module « produit».
\newline
Le présent mémoire expose les résultats de notre travail durant notre stage. Il comporte cinq chapitres organisés suivant le processus de développement choisi:
\begin{itemize}
\item L’objet du premier chapitre est de cerner le projet dans son contexte général, à travers la présentation de la société Thinline, pour décrire ensuite la motivation du projet et les objectifs visés. La dernière section du chapitre sera consacrée à la conduite et la planification du projet.
\item Le deuxième chapitre et le troisième chapitre englobent respectivement l’analyse et la spécification des besoins, à savoir l'étude fonctionnelle, dont l’objectif est de capturer les besoins fonctionnels et opérationnels du système futur, aussi bien que l'étude technique, qui englobe l’architecture physique et logicielle ainsi que les outils les plus appropriés pour ce type d'application.
\item Au niveau du quatrième chapitre, nous explicitons les phases d’analyse et de conception de la plateforme. Nous utilisons les différents diagrammes uml pour modéliser les fonctionnalités de l’application.
\item Le cinquième et dernier chapitre décrit d’une façon détaillée, la phase de mise en œuvre du projet, à travers quelques interfaces réalisées.
\end{itemize}

\noindent Le mémoire se termine par une conclusion générale qui présente le bilan du travail réalisé et ses principales perceptives.





\clearpage

\end{onehalfspace}