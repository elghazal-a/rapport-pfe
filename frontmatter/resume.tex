
\chapter*{Résumé}
\begin{singlespace}
\initial {L}e présent mémoire constitue le fruit d’un travail de quatre mois, réalisé dans le cadre de notre projet de fin d’études, et effectué au sein de Thinline. Le projet a pour objectif le développement d'un  (Customer Relationship Management) pour les laboratoires pharmaceutiques.
\newline
\newline
Il s'agit en effet d'étudier, de concevoir et de réaliser une plate-forme automatisée de gestion de flotte. Pour arriver à cette fin, nous avons pour mission de développer trois modules essentiels pour tout développement future:

    \begin{itemize}
        \item Le module annuaire.
        \item Le module calendrier.
        \item Le module produit.
    \end{itemize}

\noindent Pour bien mener ce projet, et étant donnée son volet technologique important, nous avons choisit de suivre la démarche de conduite de projet , une démarche qui a fait ses preuves dans le domaine des projets informatiques.
\newline
\newline
Durant ce projet, nous avions pour mission dans un premier temps de découvrir le métier,  cerner le sujet, étudier sa faisabilité et définir le cahier des charges, ainsi que rédiger le dossier de spécifications fonctionnelles aussi bien que technique. Ensuite nous avons entamé l’analyse approfondie et la conception de notre projet, nous avons par la suite élaboré les différents diagrammes , à savoir les diagrammes des cas d’utilisation, les diagrammes de séquences puis les diagrammes de classes, et finalement nous avons passé à l’implémentation, le test et le déploiement de l’application.
\vfill{\textbf{Mots clés:} CRM, CRM pharmaceutique, 2TUP, SaaS, Symfony2, AngularJs.}
\end{singlespace}